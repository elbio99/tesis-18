\chapter{Conclusiones y lineamientos futuros}\label{chap:conclusiones}
Es este capítulo se van a exponer, desde un punto de vista personal y subjetivo, las lecciones aprendidas durante
la realización de este trabajo final. Además se propondrán posibles evoluciones o mejoras que se podrían aplicar sobre los
desarrollos realizados.

\section{Conclusiones}
En base a la investigación y el desarrollo que se realizo podemos exponer lo siguientes items:
\begin{enumerate}
\item Para realizar un reconocimiento óptimo de patrones se necesita un datasets representativo de las regiones que se desea detectar.
\item Debido al tamaño de las imágenes y que las regiones buscadas eran muy pequeña en proporción a la misma, se demostró que para obtener regiones candidatas a partir de \textit{regions proposal} se debe realizar recortes (crops) de acuerdo al tamaño de la región buscada.
\item Se observo que a causa de que  la mayoría de los \textit{feature vector} calculados son de etiquetas negativas, es decir que no caen en la 
región buscada, al realizar el entrenamiento de los datos se obtenía un modelo con overfitting. A continuación se empleo la siguiente regla 3 
negativos cada 1 positivo; con esto no solo evitamos el overfitting del modelo sino también mejoramos el tiempo de computo en el procesamiento. 
\item Se emplearon diferentes algoritmos de búsqueda de parámetros \textit{GridSearch} y \textit{RamdomizeSeach}, el tiempo observado de procesamiento mas óptimo fue por medio de \textit{RamdomizeSeach}.
\item Se demostró experimentalmente que es posible reconocer patrones en una imagen satelital por medio de técnicas de \ac{va} y \ac{ml}.
\item Se demostró que las redes neuronales profundas obtienen una buena generalización de la imagen para problemas en reconocimiento de patrones.
\end{enumerate}


\section{Lineamientos futuros} 
Tomando como base las lecciones aprendidas en el desarrollo de la tesis; las posibles lineas de trabajo a futuro que se podrían seguir son:
\begin{enumerate}
 \item A partir del bajo rendimiento en el calculo de los datos surge la posibilidad de implementar procesamiento en base a GPU para el cómputo 
en paralelo de los datos.
 \item Es posible aplicar diferentes arquitectura de las redes neuronales para la extracción de los vectores de característica; de esta manera 
determinar cual de las redes tiene mayor generalización en los datos para el tipo de problemática que estamos tratando.
 \item Para el cálculo de regiones propuestas es posible realizar \textit{custom regions proposal} para lograr reconocer regiones específicas de una 
imagen sin realizar recortes en la misma.
 \item Conseguir implementar el modelo un una placa de vuelo para determinar el rendimiento y consumo.
\end{enumerate}
