\chapter{Conclusiones y lineamientos futuros}\label{chap:conclusiones}
En este apartado se expone las conclusiones y lineamientos futuros obtenidos a partir de la investigación realizada en la presente tesis. 

\section{Conclusiones}
En el desarrollo de la presente tesis se abordo el problema del reconocimiento de patrones sobre una imagen satelitales. EL estudio derivo en evaluar diferentes bandas a utilizar, pre procesamiento de imágenes ademas de la selección de diversas herramientas informáticas para su desarrollo y conclusión.

Como base de lo expresado anteriormente se expone las siguientes conclusiones obtenidas por esta investigación:
\begin{enumerate}
\item El reconocimiento de patrones y las técnicas de aprendizaje automático nos dan diversas ventajas a la hora de optimizar nuestro procesos; en nuestro caso la detección de regiones. Para poder realizar un modelo eficiente se debe tener en cuentas numerosos factores en el cual uno de los mas importante son los datos que vamos a utilizar; estos datos deber ser representativo del modelo que deseamos crear.

\item Cuando se trabaja con imágenes de gran tamaño se deben tener en cuenta el tamaño de regiones que se quiere captar ademas del hardware necesario para poder realizar el computo de los datos.

\item Otras de las observaciones que se detecto es que las muestra de los datos deben estar balanceadas, es decir tener las mismas cantidad de datos tanto negativos como positivos para poder discriminar de manera adecuada las clases que deseamos detectar.

\item Unos de los principales inconvenientes detectados fue el \textit{overfitting} de los datos; como se menciona en el punto anterior al tener clase desbalanceadas los modelos entrenados no hacian una buena generalización de los datos. Para evitar esto se trabajo con técnicas de \textit{cross validation} y resampleo de los datos.

%\item Se demostró experimentalmente que es posible reconocer patrones en una imagen satelital por medio de técnicas de \ac{va} y \ac{ml}.
\item En base a las experimentaciones realizadas podemos afirmar que aplicar redes neuronales pre entrenadas para la extracción de característica se obtiene una correcta generalización de los datos y son muy eficiente para problemas de clasificación de imágenes.

\end{enumerate}


\section{Lineamientos futuros} 
Tomando como base las lecciones aprendidas en el desarrollo de la tesis; las posibles lineas de trabajo a futuro que se podrían seguir son:
\begin{enumerate}
 \item A partir del bajo rendimiento en el calculo de los datos surge la posibilidad de implementar procesamiento en base a GPU para el cómputo 
en paralelo de los datos.
 \item Es posible aplicar diferentes arquitectura de las redes neuronales para la extracción de los vectores de característica; de esta manera 
determinar cual de las redes tiene mayor generalización en los datos para el tipo de problemática que estamos tratando.
 \item Para el cálculo de regiones propuestas es posible realizar \textit{custom regions proposal} para lograr reconocer regiones específicas de una 
imagen sin realizar recortes en la misma.
 \item Conseguir implementar el modelo un una placa de vuelo para determinar el rendimiento y consumo.
\end{enumerate}
