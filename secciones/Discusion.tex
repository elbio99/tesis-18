\chapter{Conclusiones y lineamientos futuros}\label{chap:conclusiones}


\section{Conclusiones}
En el desarrollo de la presente tesis se abordo un problema de detección buscando patrones de interés, laguna Mar Chiquita y golfo de San Matías, dentro de una imagen satelital. Para la construcción de la solución se utilizaron diferentes técnicas para el pre-procesamiento de la imagen, redes neuronales para la extracción de característica y algoritmos de clasificación. 

Como resultado de todos los experimentos realizados se puede exponer las siguientes conclusiones:

\begin{enumerate}
\item Utilizar técnicas de aprendizaje automático en el proceso de automatización de tareas, como en este caso la detección de patrones en una imagen satelital, nos brinda una fuente de información extra para poder realizar el proceso de calibración/detección que se busca en el satélite. 

\item Al trabajar con imágenes satelitales en donde las resoluciones de la misma son grandes, se debe tener en cuenta el tamaño de la regiones de interés que se quiere detectar.

\item Uno de los puntos importantes es el tipo del hardware que vamos a utilizar, el cómputo de los vectores de característica utilizando redes neuronales es costoso en términos de memoria y velocidad, es por este motivo que en la mayoría de los casos de uso utilizan GPU para el procesamiento.

\item En base a las experimentaciones realizadas podemos afirmar que aplicar redes neuronales pre-entrenadas para la extracción de característica se obtiene una correcta generalización de los datos y son muy eficiente para problemas de clasificación de imágenes.

\item Unos de los principales inconvenientes detectados fue el \textit{overfitting} de los datos, esto se debió al desbalanceo de clases ya que se recolectó muy pocas muestras positivas, en la mayoría de los casos debido a las nubes que interceptaban la región de interés buscada. Se soluciono este problema usando técnicas de cross-validation.

\end{enumerate}

El costo computacional en término de capacidad de operaciones hace que utilizar técnicas de \ac{cv} sea  muy costoso de aplicar en vuelo. Unos de los principales inconvenientes son la limitaciones de energía que posee un micro-nano-satélite, es por esto que se torna inviable en términos de energía y capacidad de almacenamiento poder implementar esta tecnología. No obstante pudimos ver que herramientas como \ac{ml} y \ac{cv} ayudan de manera eficiente a detectar elemento sobre una imagen satelital automatizando los procesos.



\section{Lineamientos futuros}\label{lineafuturas}
Tomando como base las lecciones aprendidas en el desarrollo de la tesis, las posibles lineas de trabajo e investigaciones a futuro son:
\begin{enumerate}
 \item A partir del bajo rendimiento en el cálculo de los datos surge la posibilidad de implementar procesamiento en base a GPU para el cómputo en paralelo de los las imágenes.
 \item Es posible aplicar diferentes arquitectura de las redes neuronales para la extracción de los vectores de característica; de esta manera determinar cual de las redes tiene mayor generalización en los datos para el tipo de problemática que estamos tratando.
 \item Evaluar nuevas herramientas de extracción de características y computar redes mucho más livianas en términos de consumo de energía.
\end{enumerate}