\chapter*{Resumen}
\label{chap:resumen}

El campo del aprendizaje automático en el reconocimiento de patrones es un área de estudio que en la actualidad es de gran interés para la comunidad científica como así también para la sociedad; hoy en día vemos como los avances de las tecnologías impacta en una gran proporción al mundo en el que vivimos desde aplicaciones de ocio en el celular hasta herramientas que proporcionan mejora en los procesos automatizando de manera eficiente los datos.

En la industria satelital existe gran cantidad de información que aún no fueron explotadas, en épocas pasadas esto se debía a los grandes costos que tienen el desarrollo de un satélite como también el acceso y procesado de los datos. En la actualidad no es así, esto se debió gracias a los desarrollos de pequeños satélites \textit{CubeSat} que permitieron abaratar costos y por otra parte el avance del poder de computo.

Tomando como ejemplo lo expuesto anteriormente el objetivo de esta tesis es realizar una evaluación experimental haciendo uso  de herramientas de Computer Vision y Machine Learning para la detección de patrones en imágenes satelitales. 
Por otra parte, debido a la gran varianza de las imágenes y la gran cantidad de información que puede entregar una imagen satelital con diferentes bandas se realizo una manipulación exhaustiva de los datos con el fin de optimizar las predicciones de los algoritmos entrenados. 

La información extraída de una imagen satelital puede ser de gran ayuda para optimizar el almacenamiento de un CubeSat, servir de apoyo en el mecanismo de apuntamiento del satélite como también un punto de partida para la realización de nuevas aplicaciones usando estos nuevos enfoques sobre imágenes satelitales.

Para concluir se realizaron diversos entrenamientos y cálculos de métricas para obtener una evaluación cualitativa de los modelos creados con el fin de obtener una conclusión del baseline propuesto en esta tesis. Los resultados obtenidos nos hace ver que los nuevos enfoques desarrollados en el área pueden sacar provecho en el ambiente satelital haciendo mas efectivo y automático los procesos de extracción de características de interés de una imagen satelital.

\textbf{Palabras clave: CubeSat,Computer Vision, Machine Learning, teledeteccion}

