\chapter*{Resumen}
\label{chap:resumen}

En el campo del aprendizaje automático en el reconocimiento de patrones es un área de estudio que en la actualidad es de gran interés para la comunidad científica como así también en la sociedad; hoy en día vemos como los avances de las tecnologías impacta en una gran proporción al mundo en el que vivimos desde aplicaciones de ocio que las tenemos en un celular como también herramientas para la mejora de procesos.

En la industria satelital existe gran cantidad de información que aún no fueron explotadas, en épocas pasadas esto se debía a los grandes costos que tienen el desarrollo de un satélite como también el acceso y procesado de los datos. En la actualidad no es así, esto se debió gracias al desarrollos de pequeños satélites \textit{CubeSat} que permitieron abaratar estos costos y el avance del poder de computo.

Tomando como partida lo expuesto anteriormente el objetivo de esta tesis es realizar una evaluación experimental haciendo uso  de herramientas de Visión Artificial, Aprendizaje Automático en la detección de patrones sobre imágenes satelitales. Esta información extraída puede ser de gran ayuda a mejorar el almacenamiento, servir de apoyo como mecanismo de apuntamiento del satélite, dando también un punto de partida para la realización de nuevas aplicaciones usando estos nuevos enfoques sobre imágenes satelitales.

Debido a la gran varianza de los datos y también a la gran cantidad de información que puede entregar una imagen satelital con diferentes bandas se realizo una manipulación exhaustiva de los datos con el fin de optimizar las predicciones de los algoritmos entrenados. 

Para concluir se realizaron diversos entrenamientos y cálculos de métricas para obtener una evaluación cualitativa de los modelos creados con el fin de obtener una conclusión del baseline propuesto. Los resultados que se obtuvieron nos hace ver que estos nuevos enfoques desarrollados se pueden sacar provecho en el ambiente satelital haciendo mas efectivo y automático los procesos.

%%%% Texto (no menos de 200 palabras)

\textbf{Palabras clave: CubeSat, Visión Artificial, Detección de Patrones, Aprendizaje Automático, teledeteccion}

